%% Introduction / first chapter.

\chapter{Introduzione}

L'obiettivo di Guglielmo Marconi negli anni 10 del '900 non era più, come precedentemente, la conquista della distanza, bensì la conquista della copertura del massimo territorio. 

In questa pubblicazione descriveremo la storia della rete denza fili nella penisola iberica.

%%%
\section{How to Use this Book}

This is a book about itself --- it's about how it was written using
the templates and tools described in 
the next few chapters. These are the main ways you can use this material:

\begin{itemize}
    \item You can read it. It shouldn't take long, it outlines the process
    used to make this eBook end to end, and then you can decide if this is something you want to try.
    \item You can use it as an instruction manual, learning and following some of the procedures step-by-step to make your own book.
    \item You can use it as a template. All of the source files used to make this book
    are freely available in GitHub at {\small \url{https://github.com/dwiddows/ebookbook}} and Overleaf.
    The source files are laid out in a way that should make it easy to clone the project and adapt it for your own book.
\end{itemize}

It follows that you could recreate this eBook for yourself, following just the process
described in the book: which is basically to clone the GitHub project as a template, build
the project, send the output HTML document through an ePub converter, and send this to
your eReader device.

So why would anyone buy a book if it's free? Because
anyone who reads the steps above and with enough familiarity to think
``git clone \ldots check dependencies \ldots  build.sh \ldots 
check dependencies again \ldots build.sh \ldots  \ldots yeah, alright'' will expect it to take
more than a few minutes, hopefully less than an hour, and price their own time realistically.
If you want to read this book for free on your eReader,
compiling from source is the way to go about it. Or if you want to just click `buy' now and send the author most of the
\$2.99 price tag, please go ahead, and thank you!

Either way, if you're not put off by \latex and \smalltt{git}
commands, keep reading. I hope the book is useful to you, and wish you 
all the best of luck and persistence writing your book!
